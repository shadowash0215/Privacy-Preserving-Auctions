\pagenumbering{arabic}
\section{Architecture and Protocol}
\subsection{High-level Description of the Protocol}
有别于博弈论中的二价拍卖机制,我们认为拍卖师是一个不可信的实体,因此我们需要引入一个新的实体来保护竞拍人的隐私,其被称为拍卖发行人(auction issuer)。拍卖发行人负责设计出输出所需结果的程序,这一程序能够输出出价最高的竞拍人编号以及第二高的价格,其余均不输出。但同样,其也不能获取到竞拍人的估值信息。而下面的协议则保证了除非拍卖师与拍卖发行人合谋,否则竞拍人的隐私将不会被泄露。

\begin{enumerate}
    \item 拍卖师公布有关拍卖的信息,包括拍卖的物品、拍卖的时间、拍卖的规则以及拍卖发行人。
    \item 竞拍人向拍卖师提交加密后的出价,拍卖发行人可以解密其中部分信息,但没法获得竞拍人的估值信息。
    \item 拍卖发行人生成程序。这一程序由电路进行描述,即使用导线以及逻辑门形成,而后拍卖发行人对其进行混淆。拍卖师将出价发送给拍卖发行人,拍卖发行人通过解密获得竞拍人的“混淆输入”,而后将其与混淆电路、“混淆输出”对照表一同发送给拍卖师。
    \item 拍卖师通过拍卖发行人发送的信息,将竞拍人的“混淆输入”输入到混淆电路中,获得“混淆输出”,而后将其与拍卖发行人发送的“混淆输出”对照表进行对照,获得出价最高的竞拍人编号以及第二高的价格。
\end{enumerate}

\subsection {Cryptographic Tools}

信息传输涉及到伪随机数的生成以及不经意传输协议,我们选择使用 AES 算法进行伪随机数的生成,并且使用了 Fixed-key Block Cipher 模式优化了乱码电路的生成。而在不经意传输协议中,因为涉及到三方的信息传输,所以需要使用代理不经意传输。但以下先介绍只涉及两方的乱码电路的生成以及不经意传输,因为三方协议的实现是基于三组两方协议的组合。

\subsubsection{Secure Function Evaluation for Two Parties}

这里采取的与姚氏乱码电路有所不同,我们认为只有 $A$ 提供输入 $x$,而 $B$ 提供程序 $f$,最终由 $A$ 计算出输出 $f(x)$.

\begin{enumerate}
    \item \textbf{电路加密}:$B$ 会为电路中的每根导线 $i$ 赋予一个长度为 $2$ 的随机数元组 $(W_i^0, W_i^1)$,分别对应这个导线的 $0$ 和 $1$ 两种状态,这一随机数元组中的值将会作为伪随机数生成的密钥的输入组成。而将这个导线的当前状态记为 $b_i \in \{0, 1\}$,还需要将这一状态通过函数 $\pi_i$ 随机映射到 $0$ 或 $1$,即 $c_i = \pi_i(b_i)$,这一映射函数 $\pi_i$ 由 $B$ 生成。最终,$B$ 将 $<W_i^{b_i}, c_i>$ 作为这一导线的“混淆值”。而对于两输入逻辑门 $g$,其接受两个输入导线 $i$ 和 $j$,其输出导线为 $k$,显然 $b_k = g(b_i, b_j)$,而 $c_k = \pi_k(b_k)$,但 $B$ 还需要为门 $g$ 生成相应的“混淆表” $T_g$,因为 $A$ 只能获取“混淆输入”而不是实际输入,$T_g$ 就是为了能让 $A$ 在不知道实际输入的情况下计算出“混淆输出”,具体计算方式如下: \[
        c_i, c_j: \ \left<(W_k^{g(b_i, b_j)}, c_k) \oplus F_{W_i^{b_i}} (c_i) \oplus F_{W_j^{b_j}} (c_j)\right> 
    \]

    其中 $F$ 是伪随机数生成函数,$0 \leqslant c_i, c_j \leqslant 1$. 因为 $c_i, c_j$ 是“混淆值”的最后一位,所以 $A$ 可以依此的定位到“混淆表”中的对应项 $T_g (c_i, c_j)$,然后将其与 $F_{W_i^{b_i}} (c_i) \oplus F_{W_j^{b_j}} (c_j)$ 进行异或,就可以得到导线 $k$ 的“混淆输出” $(W_k^{g(b_i, b_j)}, c_k)$. 

    而 Fixed-key Block Cipher 模式产生伪随机数基础组成部分是流密码与异或。我们采用了 AES 加密,key 为 128 位,具体的计算表达式为 \begin{align}
        K & {} = 2 X^a \oplus 4 X^b \oplus T \\
        Enc(X^a, X^b, T, X^c) & {} = \pi(K) \oplus K \oplus X^c 
    \end{align}
    其中 $T$ 指逻辑门的标识符,这里我们用其输出导线的序号来表示。$\pi$ 就是 Fixed-key Block Cipher 模式的加密函数,$X^a, X^b, X^c$ 分别是输入导线 $i, j$ 以及输出导线 $k$ 的“混淆值”。所以可以发现,Fixed-key Block Cipher 模式的伪随机数生成并不像上面计算“混淆表”那样是分开产生两个伪随机数,而是将两个输入导线的“混淆值”与输出导线的序号一同输入到 AES 加密中。

    \item \textbf{编码输入}:$B$ 生成好这样的电路后,首先将电路结构、“混淆表”、“混淆输出对照表” 发送给 $A$,然后 $A$ 获取与自己输入 $x$ 对应的“混淆输入”,即 $A$ 需要将 $x$ 通过某种方式处理为二进制形式,而后将每一位与一根输入导线对应,并通过不经意传输获得对应的“混淆输入”。两方的不经意传输有基于 RSA 的协议,但我们这里采取的是基于 ElGamal 的协议,具体的通讯过程如下:\begin{enumerate}
        \item \textbf{初始化}:$A$ 和 $B$ 选择在一个由生成元 $g$ 产生的大循环群 $G_g$ 上进行通信,其上的离散对数问题应当是困难的,并且选择了其中的一个元素 $c \in G_g$ 作为共同知识。
        \item \textbf{发起请求}:$A$ 选择一个随机数 $0 < r < \lvert G_g \rvert$,然后计算出 $PK_{\sigma} = g^r$ 以及 $PK_{1 - \sigma} = c/PK_{\sigma}$,而后将 $PK_0$ 发送给 $B$。这里的 $\sigma$ 是 $A$ 的输入,$PK_i, i \in \{0, 1\}$ 是 $A$ 的公钥,$r$ 是相对于 $PK_{\sigma}$ 的私钥。
        \item \textbf{发送信息}:$B$ 计算出 $PK_1 = c/PK_0$,然后将加密后的消息 $E_{PK_0}(m_0), E_{PK_1}(m_1)$ 发送给 $A$,其中 $m_i$ 是导线对应的“混淆值”。我们采用的加密算法参照了 Nigel Smart's "Cryptography Made Simple",$B$ 选择一个随机数 $0 < k < \lvert G_g \rvert$,并且计算出 $PK_i^k, i \in {0, 1}$,然后计算其哈希 $h_i = H(PK_i^k)$,获得一个与对应加密消息相同长度的哈希值后与消息异或,最终得到 $E_{PK_i}(m_i)$。为了保证 $A$ 能够解密,$B$ 还需要将 $g^k$ 一同发送给 $A$。即最终需要发送 \[
            (GK, e_0, e_1) = (g^k, m_0 \oplus H(PK_0^k), m_1 \oplus H(PK_1^k))
        \]
        \item \textbf{解密}:$A$ 收到消息后,会去解密 $e_{\sigma}$ 以得到 $m_{\sigma}$,具体的操作是使用对应的哈希算法去计算 $H(GK^r)$,而后异或 $m_{\sigma}$,即 \[
            e_{\sigma} \oplus H(GK^r) = m_{\sigma} \oplus H(g^{kr}) \oplus H(g^{kr}) = m_{\sigma}
        \]
    \end{enumerate}

    \item \textbf{计算输出}:$A$ 通过上述的不经意传输协议获得了全部的“混淆输入”,依据电路的结构以及“混淆表”计算出“混淆输出”,并与“混淆输出对照表”对照,最终得到输出 $f(x)$.


\end{enumerate}

\subsubsection{Secure Function Evaluation for Auctions}

而最终的隐私保护拍卖的实现就是在上述的两方协议的基础上进行的,以下是作出相应调整的部分:

\begin{itemize}
    \item 因为二价拍卖所需要的函数是确定的,所以我们对其实现了模块化。具体而言需要进行两次比较,第一次比较得到最大值后得到对应的竞拍人编号,然后将这一最大值置零再进行比较,得到第二大值。
    \item 两方通讯的时候我们提到 $A$ 会选择解密 $e_{\sigma}$ 以得到 $m_{\sigma}$,此处的 $A$ 相当于竞拍人。但是当我们使用代理不经意传输协议的时候,拍卖师作为代理并不能知道竞拍人的输入,所以拍卖师需要去尝试解密两个消息。而为了选择出哪条是需要的输入,就需要拍卖发行人在发送消息的时候附带纠错码,我们的处理是在消息头部附带了消息的 SHA-256 哈希值,而后拍卖师会尝试解密两个消息,然后计算出哈希值,如果其中一个消息的哈希值与消息头部的哈希值相同,那么就说明这个消息是需要的输入。
    \item 我们允许了竞拍人与拍卖发行人的直接通讯以简化流程,但事实上整个流程应当是诸拍卖人将自己的信息先发送给拍卖师,拍卖师收集了足够的信息后再与拍卖发行人通讯。而需要由拍卖发行人获取的竞拍人的信息需要使用拍卖发行人的公钥加密,以保证拍卖师无法获取到竞拍人的信息。
\end{itemize}\pagenumbering{arabic}
\section{Single-minded Envy-free Problems with Limited Supply}
