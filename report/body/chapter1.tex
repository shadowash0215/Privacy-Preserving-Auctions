\pagenumbering{arabic}
\section{Problem Statement}

在课程中我们已经学习过,二价拍卖机制可以让竞拍人(bidder)按照内心的估值出价,从而达到最优的效果。但是考虑以下的情形:竞拍人 $A$ 以 $v_A$ 的价格赢下了物品,但是他需要支付的价格 $v_B$ 与 $v_A$ 非常的接近,这样的话,$A$ 难免会怀疑拍卖师(auctioneer)暗中操纵了拍卖的结果。

当然,这个问题可以通过在拍卖结束时展示价格来解决,但考虑连续竞拍的情况,竞拍人 $A$ 以 $v_{A1}$ 的价格赢下了物品,仅需支付 $v_{B1}$;第二次竞拍同样的一个物品时,拍卖师却宣布存在一个保留价格 $r$,$r$ 又与 $v_{A1}$ 非常的接近,$A$ 会怀疑拍卖师在第一次拍卖中获取了他的估值信息,从而设置了保留价格。

所以二价拍卖机制确实是一个很好的机制,但我们同样需要考虑竞拍人的隐私问题,以防止拍卖中的欺诈行为。以下我们复现了这篇文章 \cite{naor1999privacy} 关于隐私拍卖机制的设计。
